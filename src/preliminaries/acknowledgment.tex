\chapter*{Acknowledgment}
\addcontentsline{toc}{chapter}{Acknowledgment}

\begin{center}
{\em Trust in the \textsc{Lord} with all thine heart; and lean not unto thine own understanding. In all thy ways acknowledge him, and he shall direct thy paths.}

(Proverbs 3:5-6)
\end{center}

I must first acknowledge my God through Whom all things are possible and Whose hand I see guide me each and every day.

A very near second is my wife, Alyssa, without whom this work wouldn't have been possible. I look up to her consistent example of discipleship, critical thinking, and motherhood. My children and family give me my ``reason why'' every day. Also, it is very advantageous when writing a dissertation to have a writer for a wife who can double as a personal editor.

My mom and dad, Denise and Veraun, have been my biggest supporters since day one. They are outstanding examples of hard work and dedication.

I want to acknowledge my Grandma Dianne, whose conversations about the universe are one of the reasons I wanted to be a scientist.

Thank you so much to my advisor Donna Calhoun. She has been the foundation for this work and her wisdom, experience, feedback, and encouragement have pushed me to excel. She is indeed an expert in her field and I consider myself extraordinarily lucky to have worked with her these last 3 (ish) years.

To my colleagues and mentors Michal Kopera, Grady Wright, Carsten Burstedde, Marylesa Howard, Alice Durand, Branson Stephens, Matthew Memmott, Andrew Ning and many others: thank you for mentoring and advising the next generation of scientists and engineers. I have learned volumes working with and under each of you.

Lastly, some ``official'' acknowledgements:

This work was partially funded by the National Science Foundation (NSF-DMS \#1819257) and the Boise State University School of Computing through the Graduate Assistant program. I want to thank the good folks in the BSU Research Computing Department and the C3+3 collaboration for the technical support behind the scenes working on the Borah and Falcon supercomputers.

This research used resources of the Argonne Leadership Computing Facility, a U.S. Department of Energy (DOE) Office of Science user facility at Argonne National Laboratory through a Director's Discretionary Allocation and is based on research supported by the U.S. DOE Office of Science-Advanced Scientific Computing Research Program, under Contract No. DE-AC02-06CH11357.

% Funding sources: NSF and BSU School of Computing Ph.D. program
% Compute: Borah @ BSU and Polaris @ ANL (ALCF)
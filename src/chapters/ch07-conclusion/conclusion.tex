\section{Conclusion}
\label{sec:conclusion}

This dissertation provides the motivation, derivation, and implementation of the quadtree-adaptive Hierarchical Poincar\'e-Steklov (QAHPS) method for solving elliptic \gls{pdes} on adaptive meshes.

\refchap{chap:qahps} derives and verifies the \gls{qahps} method for use on adaptively refined finite volume meshes. It is an adaptive extension to the \gls{hps} method \citep{gillman2014direct} and features 4-to-1 merging and splitting algorithms for use with \pforest \citep{burstedde2011p4est}. It was shown that the implementation is accurate to $2^{nd}$ order and the adaptive implementation provides significant speedup in comparison with the uniform implementation.

The adaptive-rebuild feature of the \gls{qahps} method was highlighted in \refchap{chap:adaptive-build}. The factorization is represented as a set of solution operators $\mathcal{S}$ that are built up through merging of local subdomains. As such, the factorization can be adapted, requiring only the updates of direct ancestors of any tagged node. In practice, this yields significant speedup over having to recompute the factorization completely when the mesh is adapted. In \refchap{chap:adaptive-build}, we showed that this still yields $2^{nd}$ order accuracy at nearly 3.5 times speedup.

In \refchap{chap:parallel}, we provided details of the parallel implementation of the \gls{qahps} method using \gls{mpi}. The partitioning of the path-indexed quadtree follows the partitioning of the leaf-indexed quadtree provided from \pforest. When performing a merge or split traversal, communication is necessary to ensure that all nodes in a family are rank-local. The communication is collective for nodes participating in the merge or split callback, which leads to global all-to-all communication at the top levels. The strong and weak scaling results indicate that portions of the algorithm scale very well, while the global communication overhead creates significant bottlenecks for other parts of the algorithm.

The EllipticForest software was compared to other codes that solve elliptic \gls{pdes} in \refchap{chap:software-and-code}. Through three case studies, it was shown that the implementation of the \gls{qahps} method into EllipticForest is competitive with codes like ThunderEgg and PETSc and even outperforms certain solvers. Even for cases where other codes are more performant than EllipticForest, it features time to solutions comparable to iterative solvers at the same effective resolution. Additionally, the times reported in \refchap{chap:software-and-code} include all stages of the \gls{qahps} method (build, upwards, and solve times). For subsequent solves, only the upwards and solve times are relevant, which are orders of magnitude smaller than other methods.
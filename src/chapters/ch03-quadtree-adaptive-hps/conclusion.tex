\section{Conclusion}
\label{sec:conclusion}

We have demonstrated a new implementation of the Hierarchical Poincaré-Steklov method on a quadtree-adaptive mesh. We outlined the key differences between a binary tree and quadtree structure, as well as the linear algebra associated with a quadtree implementation. Our novel full quadtree indexing allows for efficient storage of the data matrices needed in the HPS method.

Our numerical experiments show that the quadtree-adaptive HPS method is fast and efficient in terms of time, error, and data storage. We solved three elliptic partial differential equations to demonstrate the correctness and efficiency of our method. This kind of implementation is well-suited for an adaptive mesh framework such as \texttt{p4est}. Indeed, we have shown that the quadtree-adaptive HPS method can solve the elliptic problems to similar error with up to $17$ times speedup in the build stage and up to $57$ times speedup in the solve stage. The speedup depends on how well the mesh is able to adapt to the curvature of the solution. Memory use is of particular interest when solving elliptic PDEs with a direct method, and we have demonstrated that this is a highly efficient implementation.

A major advantage of this direct solver for elliptic problems is the ability to pre-compute the set of solution operators needed to solve the problem. Once the set of solution operators is known, a solve step can be done in linear time. This is especially powerful for time-stepping problems where one can pre-compute the solution operators, then apply them at subsequent time steps. Further development of this implementation is in progress to couple this elliptic solver to a hyperbolic finite volume solver through an operator splitting approach. In addition, a fully parallel implementation is being developed for use on large supercomputers to solve coupled hyperbolic/elliptic partial differential equations.
\section{Introduction}

Elliptic partial differential equations (PDEs) describe phenomena in the areas of physics, biology, chemistry, engineering, and others. In physics, potential fields such as gravitational fields and electromagnetic fields are modeled with elliptic PDEs (the Laplace and Poisson equations). In biology and chemistry, reaction-diffusion systems can be characterized using time-dependent elliptic operators. Engineering examples include heat and mass diffusion, as well as pressure calculations for viscous fluid systems. Scientists and engineers modeling these various phenomena must solve the associated elliptic equations using analytical and/or numerical methods.

Numerical methods for elliptic PDEs is a well-documented area of research and development. Classical methods include finite difference, finite volume, and finite element methods. More modern approaches include higher order versions of classical methods, spectral methods, and meshless approaches like particle based methods or radial basis function techniques. Many discretization approaches results in a system of linear equations that must be solved efficiently and rapidly. \damyn{[More here.]}

Modern compute resources are as diverse as they are extensive. From high-powered laptops, to desktop workstations, to clusters and supercomputers, there is no lack of available platforms for scientific modeling. With diverse hardware and software stacks, the numerical methods developed to solve the aforementioned linear systems should be flexible enough to target modern compute resources. \damyn{[More here.]}

\damyn{AMR discussion: why use AMR; different flavors of AMR; additional considerations when using AMR}
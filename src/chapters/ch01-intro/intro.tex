\section{Introduction}

Elliptic partial differential equations describe physical systems that arise in the areas of physics, biology, chemistry, engineering, and others. They model phenomena like heat and mass dispersion, electromagnetism, fluid systems, and more. Modern approaches to solving such elliptic partial differential equations (PDEs) include numerical methods that target computational resources that range from a personal laptop or desktop computer to clusters and the largest supercomputers. Making efficient use of massive compute power is vital to make advances in scientific and engineering fields.

The numerical methods employed to solve PDEs on modern platforms are non-trivial in their algorithms or implementation. Developing software that targets CPU or GPU machines 

Depending on the application, it is advantageous to use techniques that improve the accuracy of the numerical solution but come at the cost of algorithmic or software complexity. Such techniques include adaptive mesh refinement (AMR), fast methods for solving the subsequent linear systems, or a variety of discretization methods.

% Many physical systems can be described using differential equations. Examples include Newton's Laws of Motion, almost all conservation laws (mass, momentum, energy, etc.), many engineering applications, and many more. To simulate or model such systems, we often use numerical techniques to discretize and solve the corresponding problem on computers. As computational resources are finite, researchers must employ efficient algorithms to solve these problems.

% Elliptic partial differential equations are a class of PDEs that explain global and steady state phenomena. Elliptic PDEs arise in fluid dynamics, heat transfer, electromagnetism, geophysics, biology, and other application areas. Many elliptic problems can be solved efficiently, but problems on complex geometries or complicated meshes are more challenging to solve. Solution methods for elliptic PDEs is a well studied topic, with many papers, books, and courses detailing their methods. However, fast and efficient ways to solve elliptic PDEs is an ongoing field of research. Improvements on classical methods, as well as new ideas, have been recently introduced.


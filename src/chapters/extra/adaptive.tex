\section{Adaptive and Unstructured Mesh Methods for Elliptic Partial Differential Equations}
\label{sec:adaptive}

In many applications of elliptic solvers where the dynamics of the problem are localized (i.e. shock waves and fronts) or the geometry of the domain is complicated, more complex meshing is used. These types of meshes include dynamically adaptive meshes, where the mesh is refined/coarsened through time, and unstructured meshes, where the cells or elements are typically polygons or polyhedrons and not logically ordered like Cartesian grids. While solving elliptic PDEs on these types of meshes is more difficult, the methods we have talked about can be extended to work on complex meshes.

\subsection{Unstructured Meshes}

When working with unstructured meshes, the domain is broken into polygons (2D) or polyhedrons (3D). Finite difference methods are typically difficult to implement on unstructured grids, but other methods like finite volume and finite element methods are often built around these complex geometries. Just as discretization methods on unstructured grids have been developed, so have more robust solvers. For example, in \citep{luo2006p}, Luo et al. present an unstructured multigrid method for the discontinuous Galerkin method. They use a sequence of solution approximations of different polynomial orders. The ultraspherical spectral element method (ultraSEM) presented by Fortunato et al. is another highly practical extension of the HPS method. In \citep{fortunato2020ultraspherical}, they apply their ultraspherical spectral element method from \citep{olver2013fast} to the HPS method. Fortunato et al. use spectral discretization on mapped quadrilaterals and triangles and use the same merge process from the HPS method. They show that the HPS method also works well on mapped, unstructured meshes.

\subsection{Adaptive Meshes}

Hierarchical meshes, like the meshes used in the HPS method, are also well suited for adaptivity. On adaptive, hierarchical meshes like quadtrees (2D) and octrees (3D), a patch can be recursively split to a desired level of refinement. Areas of the domain that have increased dynamics or need higher refinement (say at boundaries or corners) can be resolved with finer meshes, while other areas can be of lower resolution. This adaptivity increases accuracy without significantly increasing runtime.

In \citep{popinet2015quadtree}, Popinet et al. use a quadtree approach to decompose the domain into a hierarchical grid. In locations where more refinement is necessary, they refine the mesh more. For example, in \citep{popinet2015quadtree}, they are solving the Serre-Green-Naghdi model (we will show this as an application in Section \ref{sec:applications}) which models shallow water equations. As they are looking at a tsunami traveling across the surface of the earth, refinement near the location of the wave fronts is necessary, but without having to have very dense meshing elsewhere. They show that they can implement the multigrid method on a quadtree based mesh.

Additionally, the HPS method has been modified to be used on an adaptive mesh. In \citep{geldermans2019adaptive}, Geldermans et al., they show how to use the HPS method on a quadtree of patches. They demonstrate the merge operation between patches on different levels, as well as discuss proper interpolation schemes for their choice of mesh (a Chebyshev tensor product). In addition, the global solution operator formed by this adaptive HPS method can still be applied to multiple right-hand side problems. The study of fast, direct solvers for elliptic problems on adaptively refined quadtrees and octrees is an active area of research.
